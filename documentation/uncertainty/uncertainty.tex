%\VignetteIndexEntry{faoswsFlag:A package to perform flag aggregation and much more}
%\VignetteEngine{knitr::knitr}
\documentclass[nojss]{jss}\usepackage[]{graphicx}\usepackage[]{color}
%% maxwidth is the original width if it is less than linewidth
%% otherwise use linewidth (to make sure the graphics do not exceed the margin)
\makeatletter
\def\maxwidth{ %
  \ifdim\Gin@nat@width>\linewidth
    \linewidth
  \else
    \Gin@nat@width
  \fi
}
\makeatother

\definecolor{fgcolor}{rgb}{0.345, 0.345, 0.345}
\newcommand{\hlnum}[1]{\textcolor[rgb]{0.686,0.059,0.569}{#1}}%
\newcommand{\hlstr}[1]{\textcolor[rgb]{0.192,0.494,0.8}{#1}}%
\newcommand{\hlcom}[1]{\textcolor[rgb]{0.678,0.584,0.686}{\textit{#1}}}%
\newcommand{\hlopt}[1]{\textcolor[rgb]{0,0,0}{#1}}%
\newcommand{\hlstd}[1]{\textcolor[rgb]{0.345,0.345,0.345}{#1}}%
\newcommand{\hlkwa}[1]{\textcolor[rgb]{0.161,0.373,0.58}{\textbf{#1}}}%
\newcommand{\hlkwb}[1]{\textcolor[rgb]{0.69,0.353,0.396}{#1}}%
\newcommand{\hlkwc}[1]{\textcolor[rgb]{0.333,0.667,0.333}{#1}}%
\newcommand{\hlkwd}[1]{\textcolor[rgb]{0.737,0.353,0.396}{\textbf{#1}}}%

\usepackage{framed}
\makeatletter
\newenvironment{kframe}{%
 \def\at@end@of@kframe{}%
 \ifinner\ifhmode%
  \def\at@end@of@kframe{\end{minipage}}%
  \begin{minipage}{\columnwidth}%
 \fi\fi%
 \def\FrameCommand##1{\hskip\@totalleftmargin \hskip-\fboxsep
 \colorbox{shadecolor}{##1}\hskip-\fboxsep
     % There is no \\@totalrightmargin, so:
     \hskip-\linewidth \hskip-\@totalleftmargin \hskip\columnwidth}%
 \MakeFramed {\advance\hsize-\width
   \@totalleftmargin\z@ \linewidth\hsize
   \@setminipage}}%
 {\par\unskip\endMakeFramed%
 \at@end@of@kframe}
\makeatother

\definecolor{shadecolor}{rgb}{.97, .97, .97}
\definecolor{messagecolor}{rgb}{0, 0, 0}
\definecolor{warningcolor}{rgb}{1, 0, 1}
\definecolor{errorcolor}{rgb}{1, 0, 0}
\newenvironment{knitrout}{}{} % an empty environment to be redefined in TeX

\usepackage{alltt}
\usepackage{url}
\usepackage[sc]{mathpazo}
\usepackage{geometry}
\geometry{verbose,tmargin=2.5cm,bmargin=2.5cm,lmargin=2.5cm,rmargin=2.5cm}
\setcounter{secnumdepth}{2}
\setcounter{tocdepth}{2}
\usepackage{breakurl}
\usepackage{hyperref}
\usepackage[ruled, vlined]{algorithm2e}
\usepackage{mathtools}
%% \usepackage{draftwatermark}
\usepackage{float}
\usepackage{placeins}
\usepackage{mathrsfs}
\usepackage{multirow}
%% \usepackage{mathbbm}
\DeclareMathOperator{\sgn}{sgn}
\DeclareMathOperator*{\argmax}{\arg\!\max}
%% \linespread{1.3}
\setlength{\parskip}{1em}

\title{\bf Parameterisation of Food Balance Sheet Uncertainty Distribution}

\author{Michael. C. J. Kao\\ Food and Agriculture Organization \\ of
  the United Nations}

\Plainauthor{Michael. C. J. Kao} 

\Plaintitle{Parameterisation of Food Balance Sheet Uncertainty Distribution}

\Shorttitle{Parameterisation of Food Balance Sheet Uncertainty Distribution}

\Abstract{ 

  In this paper, we briefly illustrate the use of the
  \pkg{faoswsFlag} package to parameterise the uncertainty
  distribution based on the confidence about a value.
  
  
  
  
  
}

\Keywords{Uncertainty, Distribution}
\Plainkeywords{Uncertainty, Distribution}

\Address{
  Michael. C. J. Kao\\
  Economics and Social Statistics Division (ESS)\\
  Economic and Social Development Department (ES)\\
  Food and Agriculture Organization of the United Nations (FAO)\\
  Viale delle Terme di Caracalla 00153 Rome, Italy\\
  E-mail: \email{michael.kao@fao.org}\\
  URL: \url{https://github.com/mkao006/sws_r_api/tree/master/faoswsFlag}
}
\IfFileExists{upquote.sty}{\usepackage{upquote}}{}
\begin{document}




\section{Introduction}

In preparing the Food Balance Sheet (FBS), one of the most
indispenable yet difficult operation is the balancing mechanism. Due
to the imperfection of data collection, estimation and imputation in
the real world, it is the norm that the FBS is unbalanced and does not
satisfy the equality constraint as first sight. Thus, a balancing
mechanism is essential for satisfying the equality between the demand
and supply of the FBS.

%% Give a footnote to the construction of the FBS.

%% Taken from the overview paper
In current practice, the choice of a variable as a balancing item
often reflects the availability of data (or the lack of data), rather
than a clear economic rationale and empirical evidence. It is
therefore not surprising that different SUA compilers/SUA approaches
have chosen different variables as their balancing items. USDA's
balances, for instance, use feed (and residual use) as the balancing
item, while the FBS often use food to balance supply and
demand. Conveniently, the XCBS approach often chooses whatever
variable is not explicitly available. Clearly, none of these
approaches renders a satisfactory solution to the problem and, no
matter what variable is used as the balancing item, this variable is
fraught with the measurement errors of all other variables. Given the
fact that there is no a priori reason to assume that the measurement
errors cancel out, the balancing item is bound to be the most
inaccurate variable of the balance.  Extending the logic to the Food
Balance Sheets, using food as the balancing item would therefore be
the least suitable solution.

%% Foot note to the problems.
Problems associated with the current approach motivated the research
team to seek a method in where inequality occuring in the FBS to be
allocated to various elements based on a logical reasoning rather than
arbitrary allocation.

One method to handling the problem is to balance the FBS on a
probabilistic basis. Each element is assumed to have a certain level
of uncertainty, and the allocation of the imbalance will depends on
the uncertainty of each element. That is, the smaller the uncertainty
we have with a particular element, the less we should apportion the
imbalance to that particular element. In the extreme case where we
have perfect certainty about an element, than no imbalance should be
added to the element.


In order to proceed with the probabilistic balancing mechanism,
formulation of distribution for each of the elements in the FBS is
necessary. The specification of the distributions undermines the
validity of the balancing mechanism, and thus a consistent and logical
construction of the distribution is crucial. The distributions should
reflect the underlying uncertainty associated with each elements in
order for the balancing mechanism to seek the optimal solution under
the equality constraint of the FBS.

A framework guiding the specification of the distributions and the
parameterisations is crucial. The absence of such framework generates
inconsistency and paradox, further the use of the term probability
maximisation is a disguise for a procedure which does not bear any
meaning.


The paper is organised as follow. An introduction to the method and
some brief theory will be provided. we will then provide illustrations
of how the package provide a consistent framework for parameterise the
uncertainty distribution of the FBS. 

A corresponding table representing the weights or confindence of
various data sources is assumed to have been constructed. For more
details on the construction of the flag table, please refer to the
faoswsFlag vignette.



\section{The Data}

\begin{knitrout}
\definecolor{shadecolor}{rgb}{0.969, 0.969, 0.969}\color{fgcolor}\begin{kframe}
\begin{alltt}
\hlkwd{library}\hlstd{(faoswsFlag)}
\hlkwd{library}\hlstd{(truncnorm)}

\hlcom{## Create new flag table}

\hlstd{exampleFlagTable} \hlkwb{=}
    \hlkwd{data.frame}\hlstd{(}\hlkwc{flagObservationStatus} \hlstd{=} \hlkwd{c}\hlstd{(}\hlstr{""}\hlstd{,} \hlstr{"E"}\hlstd{,} \hlstr{"I"}\hlstd{),}
               \hlkwc{flagObservationWeights} \hlstd{=} \hlkwd{c}\hlstd{(}\hlnum{1}\hlstd{,} \hlnum{0.2}\hlstd{,} \hlnum{0.6}\hlstd{))}

\hlcom{## Create artificial example for Supply and Utilisation Account}
\hlstd{exampleSUA.df} \hlkwb{=}
    \hlkwd{data.frame}\hlstd{(}\hlkwc{product} \hlstd{=} \hlkwd{c}\hlstd{(}\hlstr{"wheat"}\hlstd{,} \hlstr{"wheat flour"}\hlstd{),}
               \hlkwc{productionValue} \hlstd{=} \hlkwd{c}\hlstd{(}\hlnum{100}\hlstd{,} \hlnum{96}\hlstd{),}
               \hlkwc{productionFlag} \hlstd{=} \hlkwd{c}\hlstd{(}\hlstr{""}\hlstd{,} \hlstr{"E"}\hlstd{),}
               \hlkwc{importValue} \hlstd{=} \hlkwd{c}\hlstd{(}\hlnum{10}\hlstd{,} \hlnum{0}\hlstd{),}
               \hlkwc{importFlag} \hlstd{=} \hlkwd{c}\hlstd{(}\hlstr{""}\hlstd{,} \hlstr{""}\hlstd{),}
               \hlkwc{exportValue} \hlstd{=} \hlkwd{c}\hlstd{(}\hlnum{50}\hlstd{,} \hlnum{0}\hlstd{),}
               \hlkwc{exportFlag} \hlstd{=} \hlkwd{c}\hlstd{(}\hlstr{""}\hlstd{,} \hlstr{""}\hlstd{),}
               \hlkwc{foodValue} \hlstd{=} \hlkwd{c}\hlstd{(}\hlnum{0}\hlstd{,} \hlnum{80}\hlstd{),}
               \hlkwc{foodFlag} \hlstd{=} \hlkwd{c}\hlstd{(}\hlstr{"E"}\hlstd{,} \hlstr{"E"}\hlstd{),}
               \hlkwc{seedValue} \hlstd{=} \hlkwd{c}\hlstd{(}\hlnum{20}\hlstd{,} \hlnum{0}\hlstd{),}
               \hlkwc{seedFlag} \hlstd{=} \hlkwd{c}\hlstd{(}\hlstr{""}\hlstd{,} \hlstr{"E"}\hlstd{),}
               \hlkwc{wasteValue} \hlstd{=} \hlkwd{c}\hlstd{(}\hlnum{20}\hlstd{,} \hlnum{16}\hlstd{),}
               \hlkwc{wasteFlag} \hlstd{=} \hlkwd{c}\hlstd{(}\hlstr{"I"}\hlstd{,} \hlstr{"E"}\hlstd{),}
               \hlkwc{foodManuValue} \hlstd{=} \hlkwd{c}\hlstd{(}\hlnum{100}\hlstd{,} \hlnum{0}\hlstd{),}
               \hlkwc{foodManuFlag} \hlstd{=} \hlkwd{c}\hlstd{(}\hlstr{"E"}\hlstd{,} \hlstr{""}\hlstd{))}

\hlcom{## Create artificial example for the unbalanced Food Balance Sheet}
\hlstd{exampleFBS.df} \hlkwb{=}
    \hlkwd{data.frame}\hlstd{(}\hlkwc{product} \hlstd{=} \hlstr{"wheat and products"}\hlstd{,}
               \hlkwc{productionValue} \hlstd{=} \hlnum{220}\hlstd{,}
               \hlkwc{importValue} \hlstd{=} \hlnum{10}\hlstd{,}
               \hlkwc{exportValue} \hlstd{=} \hlnum{50}\hlstd{,}
               \hlkwc{foodValue} \hlstd{=} \hlnum{100}\hlstd{,} \hlkwc{seedValue} \hlstd{=} \hlnum{20}\hlstd{,}
               \hlkwc{wasteValue} \hlstd{=} \hlnum{40}\hlstd{,}
               \hlkwc{foodManuValue} \hlstd{=} \hlnum{100}\hlstd{)}
\end{alltt}
\end{kframe}
\end{knitrout}

\section{The Methodology}

After observing a single value, how does one imposes a probability
distribution that is consistent? In order to impose a distribution, we
will need to impose something which reflects the uncertainty of the
value we have observed.  Similar to assigning a probability to whether
it will rain tomorrow, we can assign a probability to the observed
value reflecting our knowledge, experience and other information that
is not explicitly incorporated. After assigning the probability to the
observed value, we then have sufficient information to specify the
distribution.


\subsection{Quantifying Uncertainty}

To measure a piece of information, one can use the formula of self
information which is a measure of the information content defined as
follow,

\begin{align}
  I = -ln(P) \, P \in [0, 1]
\end{align}


Where P is the probability or the confidence about the accuracy of the
value assigned to the observed value in the first place. The natural
logarithm is adopted here, but logarithm of any base can be used. This
is a measure of the uncertainty condition on the confidence we have
about the observed value. The log ensures that the uncertainty can be
added across different items. The greater the I, the larger the
associated uncertainty, that is, the lower the confidence we assign to
the particular value, the higher the uncertainty. When the confidence
is 0, or with 0\% certainty, then I is infinite or infinite
uncertainty; on the other hand, when the confidence is 1 then the the
value is known with certain.

The logarithm also ensures that the uncertainty is
additive. Essentially, the sum of the uncertainty is the log of the
products of the probabilities assigned to the values. That is, it is
the log of the joint probability assuming independence.

By assuming that the observed value is the expected value (the
expected value here refers to the value with the highest probability,
that is, the mode) and the self information as the expected
information or entropy of the distribution, the parameterisation of
any chosen distribution then follows naturally. Given the level of
uncertainty associated with each element, then regardless of the
choice of distributions, one can always parametrize the distribution
where the uncertainty is held the same. This provides a consistent
framework for specifying distributions in which the uncertainty for
each element is consistent and relative amongst all elements.

The following illustration provides an example of how this framework
can provide consistent parameterization of various distribution while
maintaining the same level of uncertainty with the value. Take the
wheat production example in the introduction section again, when we
have 80\% confidence about the observed value of 22,000 tonnes of
wheat production in Australia, then the amount of uncertainty given
the provided information can then be calculated as follow

\begin{align}
  I = -ln(0.8) \approx 0.2231
\end{align}

In order to preserve this uncertainty associated with this piece of
information, we need to preserve the entropy of the
distributions. That is, regardless which distribution we choose we
need to parameterise the distribution such that the entropy is
equivalent to the same nat of information available.

The main reason to use the entropy rather than the standard deviation
is because it is unit free and does not depends on the size of the
value. If we were to impose uncertainty between two values, then the
uncertainty associated with both value should be set respectively to
the confidence given independent of the magnitude of the value. For
example, if we we have observed 2000 tonnes of wheat production and
1000 tonnes of food while the confidence in the two value are
identical, then the balance should be 1500 tonnes of production and
food. If we were to base the uncertainty on standard deviation or
percentage of variation, then the larger value will have large
standard deviation of variation based on percentage and thus the final
value will be closer to 1000 even though we have equal confidence in
both values.

\subsection{Parameterise Distribution Given Uncertainty}

To parameterise a given distribution provided the uncertainty computed
above is a simple task.

One simply choose the distribution then set the entropy equivalent to
the uncertainty and solve for the parameters of the distribution.

That is, we solve for the following identity.

\begin{align}
  I = H
\end{align}

Where $I$ is the self-information or uncertainty computed, and $H$ is
the entropy of the chosen distribution.


Take the normal distribution for example which has the following entropy equation:

\begin{align}
  H = \frac{1}{2}ln(2\pi e\sigma^2)
\end{align}

We can see the only parameter of the normal distribution that governs
the entropy is $\sigma$ or the standard deviation. Then Substitude $H$
with $I$ and solve for $\sigma$, we obtain the following.

\begin{align}
  \sigma = \sqrt{\frac{e^{2I}}{2\pi e}} = \sqrt{\frac{e^{2(-ln(0.8))}}{2\pi e}} \approx 0.3025
\end{align}

or simply,

\begin{knitrout}
\definecolor{shadecolor}{rgb}{0.969, 0.969, 0.969}\color{fgcolor}\begin{kframe}
\begin{alltt}
\hlkwd{parameterise}\hlstd{(}\hlkwc{obsValue} \hlstd{=} \hlnum{22000}\hlstd{,} \hlkwc{selfInformation} \hlstd{=} \hlopt{-}\hlkwd{log}\hlstd{(}\hlnum{0.8}\hlstd{),} \hlkwc{distribution} \hlstd{=} \hlstr{"normal"}\hlstd{)}
\end{alltt}
\begin{verbatim}
## $mean
## [1] 22000
## 
## $sd
## [1] 0.3024634
\end{verbatim}
\end{kframe}
\end{knitrout}

That is, with the given observed value and confidence, the associated
distribution is:

\begin{align}
  W \sim N(22,000, 0.3025)
\end{align}

Following the same example, we can solve for the parameterisation of a
Cauchy distribution given the same level of uncertainty.

\begin{align}
  \gamma = e^{I - ln(4\pi)} = e^{-ln(0.8) - ln(4\pi)} \approx 0.0995
\end{align}
and,

\begin{align}
  W \sim Cauchy(22,000, 0.0995)
\end{align}

\begin{knitrout}
\definecolor{shadecolor}{rgb}{0.969, 0.969, 0.969}\color{fgcolor}\begin{kframe}
\begin{alltt}
\hlkwd{parameterise}\hlstd{(}\hlkwc{obsValue} \hlstd{=} \hlnum{22000}\hlstd{,} \hlkwc{selfInformation} \hlstd{=} \hlopt{-}\hlkwd{log}\hlstd{(}\hlnum{0.8}\hlstd{),} \hlkwc{distribution} \hlstd{=} \hlstr{"cauchy"}\hlstd{)}
\end{alltt}
\begin{verbatim}
## $location
## [1] 22000
## 
## $scale
## [1] 0.09947184
\end{verbatim}
\end{kframe}
\end{knitrout}


There are no analytical solution for the parameterisation of the
truncated normal distribution, however, a numerical solution is
provided by the package.



\begin{align}
  W \sim trN(22,000, 0.3025)
\end{align}

\begin{knitrout}
\definecolor{shadecolor}{rgb}{0.969, 0.969, 0.969}\color{fgcolor}\begin{kframe}
\begin{alltt}
\hlkwd{parameterise}\hlstd{(}\hlkwc{obsValue} \hlstd{=} \hlnum{22000}\hlstd{,}
             \hlkwc{selfInformation} \hlstd{=} \hlopt{-}\hlkwd{log}\hlstd{(}\hlnum{0.8}\hlstd{),}
             \hlkwc{distribution} \hlstd{=} \hlstr{"truncNorm"}\hlstd{)}
\end{alltt}
\begin{verbatim}
## $mean
## [1] 22000
## 
## $sd
## [1] 0.3024757
\end{verbatim}
\end{kframe}
\end{knitrout}


We can see that the standard deviation of the truncated normal is
marginally larger than the normal distribution above. This is due to
the fact to maintain the same level of uncertainty while reduced
support space, one has to increase the standard deviation.
  
  

%% Add a plot of all the distributions offered in the packag except
%% for the exponential distribution.

\section{Illustration}
Take the example data, the first step is to construct the uncertainty
of each FBS element based on the flags in the SUA.



\begin{knitrout}
\definecolor{shadecolor}{rgb}{0.969, 0.969, 0.969}\color{fgcolor}\begin{kframe}
\begin{alltt}
\hlcom{## Select all the flag columns}
\hlstd{flagColumns} \hlkwb{=} \hlkwd{grep}\hlstd{(}\hlstr{"Flag"}\hlstd{,} \hlkwd{colnames}\hlstd{(exampleSUA.df),} \hlkwc{value} \hlstd{=} \hlnum{TRUE}\hlstd{)}

\hlcom{## First we convert the flags to weights or confidence}
\hlstd{(weightsSUA.df} \hlkwb{=} \hlkwd{data.frame}\hlstd{(}\hlkwd{lapply}\hlstd{(exampleSUA.df[, flagColumns], flag2weight)))}
\end{alltt}
\begin{verbatim}
##   productionFlag importFlag exportFlag foodFlag seedFlag wasteFlag foodManuFlag
## 1           1.00          1          1     0.75     1.00      0.50         0.75
## 2           0.75          1          1     0.75     0.75      0.75         1.00
\end{verbatim}
\begin{alltt}
\hlcom{## Then we compute the self-information}
\hlstd{(selfInfoSUA.df} \hlkwb{=} \hlkwd{data.frame}\hlstd{(}\hlkwd{lapply}\hlstd{(weightsSUA.df, selfInformation)))}
\end{alltt}
\begin{verbatim}
##   productionFlag importFlag exportFlag  foodFlag  seedFlag wasteFlag
## 1      0.0000000          0          0 0.2876821 0.0000000 0.6931472
## 2      0.2876821          0          0 0.2876821 0.2876821 0.2876821
##   foodManuFlag
## 1    0.2876821
## 2    0.0000000
\end{verbatim}
\begin{alltt}
\hlcom{## Then we sum up the self-information for each element to obtain }
\hlcom{## the uncertainty of each element.}
\hlstd{totalInfo.df} \hlkwb{=} \hlkwd{data.frame}\hlstd{(}\hlkwd{lapply}\hlstd{(selfInfoSUA.df, sum))}
\end{alltt}
\end{kframe}
\end{knitrout}


To create the distribution, we simply prodive the function
`distributionise` the observered value, the total information computed
from the flag andthe desired distribution.

The function returns a list of two object. The first is the
distribution function with the parameters computed, while the second
object is a list with the corresponding values of the parameter.

\begin{knitrout}
\definecolor{shadecolor}{rgb}{0.969, 0.969, 0.969}\color{fgcolor}\begin{kframe}
\begin{alltt}
\hlcom{## Parameterise the production element with a Normal distribution}
\hlkwd{distributionise}\hlstd{(}\hlkwc{obsValue} \hlstd{= exampleFBS.df}\hlopt{$}\hlstd{productionValue,}
                \hlkwc{selfInformation} \hlstd{= totalInfo.df}\hlopt{$}\hlstd{productionFlag,}
                \hlkwc{distribution} \hlstd{=} \hlstr{"normal"}\hlstd{)}
\end{alltt}
\begin{verbatim}
## $pdf
## function (x) 
## dnorm(x, mean = mean, sd = sd)
## <environment: 0x2c22b98>
## 
## $parameters
## $parameters$mean
## [1] 220
## 
## $parameters$sd
## [1] 0.3226276
\end{verbatim}
\begin{alltt}
\hlcom{## Parameterise the production element with a Truncated Normal distribution}
\hlkwd{distributionise}\hlstd{(}\hlkwc{obsValue} \hlstd{= exampleFBS.df}\hlopt{$}\hlstd{productionValue,}
                \hlkwc{selfInformation} \hlstd{= totalInfo.df}\hlopt{$}\hlstd{productionFlag,}
                \hlkwc{distribution} \hlstd{=} \hlstr{"truncNorm"}\hlstd{)}
\end{alltt}
\begin{verbatim}
## $pdf
## function (x) 
## dtruncnorm(x, a = 0, b = Inf, mean = mean, sd = sd)
## <environment: 0x2b71480>
## 
## $parameters
## $parameters$mean
## [1] 220
## 
## $parameters$sd
## [1] 0.3226171
\end{verbatim}
\end{kframe}
\end{knitrout}



\begin{knitrout}
\definecolor{shadecolor}{rgb}{0.969, 0.969, 0.969}\color{fgcolor}\begin{kframe}
\begin{alltt}
\hlstd{productionDist} \hlkwb{=}
    \hlkwd{distributionise}\hlstd{(}\hlkwc{obsValue} \hlstd{= exampleFBS.df}\hlopt{$}\hlstd{productionValue,}
                    \hlkwc{selfInformation} \hlstd{= totalInfo.df}\hlopt{$}\hlstd{productionFlag,}
                    \hlkwc{distribution} \hlstd{=} \hlstr{"truncNorm"}\hlstd{)}


\hlstd{importDist} \hlkwb{=}
    \hlkwd{distributionise}\hlstd{(}\hlkwc{obsValue} \hlstd{= exampleFBS.df}\hlopt{$}\hlstd{importValue,}
                    \hlkwc{selfInformation} \hlstd{= totalInfo.df}\hlopt{$}\hlstd{importFlag,}
                    \hlkwc{distribution} \hlstd{=} \hlstr{"truncNorm"}\hlstd{)}


\hlstd{exportDist} \hlkwb{=}
    \hlkwd{distributionise}\hlstd{(}\hlkwc{obsValue} \hlstd{= exampleFBS.df}\hlopt{$}\hlstd{exportValue,}
                    \hlkwc{selfInformation} \hlstd{= totalInfo.df}\hlopt{$}\hlstd{exportFlag,}
                    \hlkwc{distribution} \hlstd{=} \hlstr{"truncNorm"}\hlstd{)}


\hlstd{foodDist} \hlkwb{=}
    \hlkwd{distributionise}\hlstd{(}\hlkwc{obsValue} \hlstd{= exampleFBS.df}\hlopt{$}\hlstd{foodValue,}
                    \hlkwc{selfInformation} \hlstd{= totalInfo.df}\hlopt{$}\hlstd{foodFlag,}
                    \hlkwc{distribution} \hlstd{=} \hlstr{"truncNorm"}\hlstd{)}


\hlstd{seedDist} \hlkwb{=}
    \hlkwd{distributionise}\hlstd{(}\hlkwc{obsValue} \hlstd{= exampleFBS.df}\hlopt{$}\hlstd{seedValue,}
                    \hlkwc{selfInformation} \hlstd{= totalInfo.df}\hlopt{$}\hlstd{seedFlag,}
                    \hlkwc{distribution} \hlstd{=} \hlstr{"truncNorm"}\hlstd{)}


\hlstd{wasteDist} \hlkwb{=}
    \hlkwd{distributionise}\hlstd{(}\hlkwc{obsValue} \hlstd{= exampleFBS.df}\hlopt{$}\hlstd{wasteValue,}
                    \hlkwc{selfInformation} \hlstd{= totalInfo.df}\hlopt{$}\hlstd{wasteFlag,}
                    \hlkwc{distribution} \hlstd{=} \hlstr{"truncNorm"}\hlstd{)}


\hlstd{foodManuDist} \hlkwb{=}
    \hlkwd{distributionise}\hlstd{(}\hlkwc{obsValue} \hlstd{= exampleFBS.df}\hlopt{$}\hlstd{foodManuValue,}
                    \hlkwc{selfInformation} \hlstd{= totalInfo.df}\hlopt{$}\hlstd{foodManuFlag,}
                    \hlkwc{distribution} \hlstd{=} \hlstr{"truncNorm"}\hlstd{)}


\hlstd{ll} \hlkwb{=} \hlkwa{function}\hlstd{(}\hlkwc{x}\hlstd{)\{}
    \hlopt{-}\hlkwd{log}\hlstd{(productionDist}\hlopt{$}\hlkwd{pdf}\hlstd{(x[}\hlnum{1}\hlstd{]))} \hlopt{-}
        \hlkwd{log}\hlstd{(importDist}\hlopt{$}\hlkwd{pdf}\hlstd{(x[}\hlnum{2}\hlstd{]))} \hlopt{-}
            \hlkwd{log}\hlstd{(exportDist}\hlopt{$}\hlkwd{pdf}\hlstd{(x[}\hlnum{3}\hlstd{]))} \hlopt{-}
                \hlkwd{log}\hlstd{(foodDist}\hlopt{$}\hlkwd{pdf}\hlstd{(x[}\hlnum{4}\hlstd{]))} \hlopt{-}
                    \hlkwd{log}\hlstd{(seedDist}\hlopt{$}\hlkwd{pdf}\hlstd{(x[}\hlnum{5}\hlstd{]))} \hlopt{-}
                        \hlkwd{log}\hlstd{(wasteDist}\hlopt{$}\hlkwd{pdf}\hlstd{(x[}\hlnum{6}\hlstd{]))} \hlopt{-}
                            \hlkwd{log}\hlstd{(foodManuDist}\hlopt{$}\hlkwd{pdf}\hlstd{(x[}\hlnum{7}\hlstd{]))}
\hlstd{\}}

\hlstd{constraint} \hlkwb{=} \hlkwa{function}\hlstd{(}\hlkwc{x}\hlstd{)\{}
    \hlstd{x[}\hlnum{1}\hlstd{]} \hlopt{+} \hlstd{x[}\hlnum{2}\hlstd{]} \hlopt{+} \hlstd{x[}\hlnum{3}\hlstd{]} \hlopt{-} \hlstd{x[}\hlnum{4}\hlstd{]} \hlopt{-} \hlstd{x[}\hlnum{5}\hlstd{]} \hlopt{-} \hlstd{x[}\hlnum{6}\hlstd{]} \hlopt{-} \hlstd{x[}\hlnum{7}\hlstd{]}
\hlstd{\}}

\hlkwd{library}\hlstd{(Rsolnp)}
\hlcom{## Double check this solution, the resulting likelihood is infinite}
\hlstd{balancedFBS} \hlkwb{=}
    \hlkwd{solnp}\hlstd{(}\hlkwc{pars} \hlstd{=} \hlkwd{as.numeric}\hlstd{(exampleFBS.df[,} \hlopt{-}\hlnum{1}\hlstd{]),}
          \hlkwc{fun} \hlstd{= ll,}
          \hlkwc{eqfun} \hlstd{= constraint,}
          \hlkwc{eqB} \hlstd{=} \hlnum{0}\hlstd{,}
          \hlkwc{control} \hlstd{=} \hlkwd{list}\hlstd{(}\hlkwc{tol} \hlstd{=}\hlnum{1e-10}\hlstd{))}
\end{alltt}
\begin{verbatim}
## 
## Iter: 1 fn: 1e+24	 Pars:  206.73973   9.97260  49.31507 102.73973  20.10959  40.43836 102.73973
## Iter: 2 fn: 1e+24	 Pars:  206.73973   9.97260  49.31507 102.73973  20.10959  40.43836 102.73973
## solnp--> Completed in 2 iterations
\end{verbatim}
\begin{alltt}
\hlstd{balancedFBS}\hlopt{$}\hlstd{par}
\end{alltt}
\begin{verbatim}
## [1] 206.739726   9.972603  49.315068 102.739726  20.109589  40.438356 102.739726
\end{verbatim}
\end{kframe}
\end{knitrout}

\section{Conclusion}

\end{document}
